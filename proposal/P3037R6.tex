\documentclass[10pt]{article}

\usepackage{color}
\usepackage{xcolor}
\usepackage{fullpage}
\usepackage[colorlinks, allcolors=blue]{hyperref}
\usepackage{listings}
\usepackage{parskip}
\usepackage{url}
\usepackage{courier}
\usepackage[T1]{fontenc}
% \usepackage{libertine}%% Only as example for the romans/sans fonts
\usepackage[scaled=0.85]{beramono}
\usepackage{changepage}
\usepackage{nameref}
\usepackage{enumitem}
\usepackage{pdfpages} % includepdf

% Solarized colour scheme for listings
\definecolor{solarized@base03}{HTML}{002B36}
\definecolor{solarized@base02}{HTML}{073642}
\definecolor{solarized@base01}{HTML}{586e75}
\definecolor{solarized@base00}{HTML}{657b83}
\definecolor{solarized@base0}{HTML}{839496}
\definecolor{solarized@base1}{HTML}{93a1a1}
\definecolor{solarized@base2}{HTML}{EEE8D5}
\definecolor{solarized@base3}{HTML}{FDF6E3}
\definecolor{solarized@yellow}{HTML}{B58900}
\definecolor{solarized@orange}{HTML}{CB4B16}
\definecolor{solarized@red}{HTML}{DC322F}
\definecolor{solarized@magenta}{HTML}{D33682}
\definecolor{solarized@violet}{HTML}{6C71C4}
\definecolor{solarized@blue}{HTML}{268BD2}
\definecolor{solarized@cyan}{HTML}{2AA198}
\definecolor{solarized@green}{HTML}{859900}

\lstset{
  basicstyle=\footnotesize\ttfamily,
}

% Define C++ syntax highlighting colour scheme
\lstdefinelanguage{cpp}{
  language=C++,
  basicstyle=\footnotesize\ttfamily,
  numbers=left,
  numberstyle=\footnotesize,
  tabsize=2,
  breaklines=true,
  escapeinside={@}{@},
  numberstyle=\tiny\color{solarized@base01},
  keywordstyle=\color{solarized@green},
  stringstyle=\color{solarized@cyan}\ttfamily,
  identifierstyle=\color{solarized@blue},
  commentstyle=\color{solarized@base01},
  emphstyle=\color{solarized@red},
  frame=single,
  rulecolor=\color{solarized@base2},
  rulesepcolor=\color{solarized@base2},
  showstringspaces=false,
}

\lstdefinelanguage{diff}{
  morecomment=[f][\color{blue}]{@@},           % group identifier
  morecomment=[f][\color{red}]{-},             % deleted lines
  morecomment=[f][\color{green!50!black}]{+},  % added lines
  morecomment=[f][\color{magenta}]{---},       % diff header lines
  morecomment=[f][\color{magenta}]{+++},
}

\lstdefinelanguage{plus}{
  basicstyle=\footnotesize\ttfamily\color{green!50!black},
  emph={see,below,TypeSwitch,unspecified},
  emphstyle=\itshape
}

\lstdefinelanguage{signature}{
  basicstyle=\ttfamily\color{green!50!black},
  emph={see,below,TypeSwitch,unspecified},
  emphstyle=\itshape
}

\newcommand{\desc}[1]{\textit{#1}}
\newcommand{\requires}{\desc{Requires}}
\newcommand{\effects}{\desc{Effects}}
\newcommand{\precondition}{\desc{Precondition}}
\newcommand{\postcondition}{\desc{Postcondition}}
\newcommand{\throws}{\desc{Throws}}
\newcommand{\returns}{\desc{Returns}}
\newcommand{\remarks}{\desc{Remarks}}
\newcommand{\exceptionsafety}{\desc{Exception Safety}}

\newcommand{\fullref}[1]{\ref{#1} \nameref{#1}}

\newcommand*{\wgxxi}[1]{[\href{https://wg21.link/#1}{#1}]}
\newcommand*{\upstreamcommit}{05300d7}
\newcommand*{\forkcommit}{fb46090}

\date{}
\title{\texttt{constexpr std::shared\_ptr} and friends}

\newcommand{\emailaddressP}{paul.keir@uws.ac.uk}
\newcommand{\emailaddressH}{cpp@hanicka.net}
\newcommand{\emailP}{\href{mailto:\emailaddressP}{\emailaddressP}}
\newcommand{\emailH}{\href{mailto:\emailaddressH}{\emailaddressH}}

\begin{document}

\setlist{nosep} % No extra spaces between itemize list entries

\maketitle\vspace{-2cm}

\begin{flushright}
  \begin{tabular}{ll}
  Document \#:&P3037R6\\
  Date:       &\date{2025-05-02}\\
  Project:    &Programming Language C++\\
  Audience:   &Library Working Group (LWG)\\
  Reply-to:   &\author{Paul Keir}\\
              &\textless\emailP\textgreater\\
              &\author{Hana Dusíková}\\
              &\textless\emailH\textgreater
  \end{tabular}
\end{flushright}

{\hypersetup{linkcolor=black}
  \tableofcontents
}

\pagebreak

\section{Revision History}

\begin{itemize}
\item R6 2025-05-02
  \begin{itemize}
  \item Added discussion on implementation of atomic partial specialisations (Section \ref{sec:atom_spec})
  \item Added discussion on implementation of smart pointer adaptors (Section \ref{sec:adapt})
  \item Removed \texttt{constexpr} specifier from wording of \texttt{void**} smart pointer adaptor conversion operators following LWG meeting
  \item Added poll result from the LEWG meeting in Hagenberg
  \end{itemize}
\item R5 2025-03-27
  \begin{itemize}
  \item Changed proposal title for clarity based on LEWG feedback in Hagenberg
  \item Included support for partial specialisations of \texttt{std::atomic} for \texttt{std::shared\_ptr} and \texttt{std::weak\_ptr}
  \item Added full context for wording changes (Section \ref{sec:wording})
  \end{itemize}
\item R4 2024-10-21
  \begin{itemize}
  \item Added \texttt{constexpr} specifier to the wording
        for \texttt{std::bad\_weak\_ptr} with \cite{P3068R6}
  \item Updated ClangOz reference in motivation
  \end{itemize}
\item R3 2024-09-03
  \begin{itemize}
  \item Removed \texttt{constexpr} specifier from
        \texttt{reinterpret\_pointer\_cast} (Section \ref{sec:mnnbs})
  \item Added references to \cite{P3309R3} and \cite{P3068R6}
  \item Added details of a second implementation based on libc++
  \end{itemize}
\item R2 2024-05-24
  \begin{itemize}
  \item Added wording
  \item Removed \texttt{constexpr} specifier from some
        functions (Section \ref{sec:mnnbs})
  \item Removed SG7 from Audience (post 2024 Spring meeting in Tokyo)
  \end{itemize}
\item R1 2024-03-05
  \begin{itemize}
  \item Added a motivating example
  \item Included libc++ \& MSVC STL in atomic operation considerations
  \end{itemize}
\item R0 2023-11-06
  \begin{itemize}
  \item Original Proposal
  \end{itemize}
\end{itemize}

\section{Introduction}

Since the adoption of \cite{P0784R7} in C++20, constant expressions can include
dynamic memory allocation; yet support for smart pointers extends only to
\texttt{std::unique\_ptr} (since \cite{P2273R3} in C++23). As at runtime, smart
pointers can encourage hygienic memory management during constant evaluation;
and with no remaining technical obstacles, parity between runtime and
compile-time support for smart pointers should reflect the increased maturity
of language support for constant expression evaluation.  We therefore propose
that \texttt{std::shared\_ptr} and appropriate classes and template member
functions from \wgxxi{smartptr}, as well as smart pointer specialisations
of \texttt{std::atomic} from \wgxxi{util.smartptr.atomic} permit
\texttt{constexpr} evaluation.

\section{Motivation and Scope}

It is convenient when the same C++ code can be deployed both at runtime and
compile time. Our recent project investigates performance scaling of
\emph{parallel} constant expression evaluation in an experimental Clang
compiler \cite{ClangOz}. As well as C++17 parallel algorithms, a prototype
\texttt{constexpr} implementation of the Khronos SYCL API was utilised, where a
SYCL \texttt{buffer} class abstracts over device and/or host memory. In the
simplified code excerpt below, the \texttt{std::shared\_ptr} data member
ensures memory is properly deallocated upon the \texttt{buffer}'s destruction,
according to its owner status. This is a common approach for runtime code, and
a \texttt{constexpr std::shared\_ptr} class implementation helpfully bypasses
thoughts of raw pointers and preprocessor macros. The impact of adding
\texttt{constexpr} functionality to the implementation is thus
minimised.

\begin{lstlisting}[language=cpp]
template <class T, int dims = 1>
struct buffer
{
  constexpr buffer(const range<dims> &r)
    : range_{ r }, data_{ new T[r.size()], [this](auto* p){ delete [] p; } } { }

  constexpr buffer(T* hostData, const range<dims>& r)
    : range_{ r }, data_{ hostData, [](auto){} } { }

  const range<dims> range_{};
  std::shared_ptr<T[]> data_{};
};
\end{lstlisting}

Adopted C++26 proposal \cite{P2738R1} facilitates a straightforward
implementation of comprehensive \texttt{constexpr} support for
\texttt{std::shared\_ptr}, allowing the \texttt{get\_deleter} member function
to operate, given the type erasure required within the
\texttt{std::shared\_ptr} unary class template. We furthermore propose that the
relational operators of \texttt{std::unique\_ptr}, which can legally operate on
pointers originating from a single allocation during constant evaluation,
should also adopt the \texttt{constexpr} specifier.

As with C++23 \texttt{constexpr} support for \texttt{std::unique\_ptr}, bumping
the value \texttt{\_\_cpp\_lib\_constexpr\_memory} is our requested feature
macro change; yet in the discussion and implementation presented here, we adopt
the macro \texttt{\_\_cpp\_lib\_constexpr\_shared\_ptr}.

We below elaborate on points which go beyond the simple addition of the
\texttt{constexpr} specifier to the relevant member functions.

\subsection{Atomic Operations}

The existing \texttt{std::shared\_ptr} class can operate within a multithreaded
runtime environment. A number of its member functions may therefore be defined
using atomic functions; so ensuring that shared state is updated correctly. In
earlier revisions of this paper, standard atomic functions were not qualified
as \texttt{constexpr}. Yet \texttt{constexpr} implementations of
\wgxxi{smartptr} classes and functions could be implemented, through
recognising that as constant expressions must be evaluated by a single thread,
execution can safely skip calls to atomic functions through the
predication of \texttt{std::is\_constant\_evaluated} (or \texttt{if
consteval}). For example, here is a modified function from the GNU C++ library
(libstdc++), called from \texttt{std::shared\_ptr::use\_count()} and elsewhere:

%The existing \texttt{std::shared\_ptr} class can operate within a multithreaded
%runtime environment. A number of its member functions may therefore be defined
%using atomic functions; so ensuring that shared state is updated correctly.
%Atomic functions are not qualified as \texttt{constexpr}; but as constant
%expressions must be evaluated by a single thread, a \texttt{constexpr}
%\texttt{std::shared\_ptr} implementation can safely skip calls to atomic
%functions through the predication of \texttt{std::is\_constant\_evaluated} (or
%\texttt{if consteval}). For example, here is a modified function from GCC's
%libstdc++, called from \texttt{std::shared\_ptr::use\_count()} and elsewhere:

\begin{lstlisting}[language=cpp]
constexpr long
_M_get_use_count() const noexcept
{
#ifdef __cpp_lib_constexpr_shared_ptr
  return std::is_constant_evaluated()
           ? _M_use_count
           : __atomic_load_n(&_M_use_count, __ATOMIC_RELAXED);
#else
  return __atomic_load_n(&_M_use_count, __ATOMIC_RELAXED);
#endif
}
\end{lstlisting}

Today, the adoption of \cite{P3309R3} has enabled \texttt{constexpr}
functionality in \texttt{std::atomic} (and \texttt{std::atomic\_ref}) for
C++26; and so code such as that above will soon be unnecessary.

\subsubsection{Atomic Partial Specialisations}
\label{sec:atom_spec}

Support for the partial specialisations of \texttt{std::atomic} for
\texttt{std::shared\_ptr} and \texttt{std::weak\_ptr} were not included in
\cite{P3309R3}. During discussions on that paper, it was proposed that
\texttt{std::atomic<std::shared\_ptr>} should be supported in
\texttt{constexpr} whenever \texttt{std::shared\_ptr} is supported in
\texttt{constexpr}. We here propose \texttt{constexpr} support for the partial
specialisations of \texttt{std::atomic} for \texttt{std::shared\_ptr} and
\texttt{std::weak\_ptr}, hitherto specified in
\wgxxi{util.smartptr.atomic.shared} and \wgxxi{util.smartptr.atomic.weak}.

Runtime implementations of \texttt{std::atomic} smart pointer specialisations,
including those within libstdc++, toggle the least
significant bit (LSB) of a pointer to a counting object (e.g.
\texttt{\_Sp\_counted\_base<>}) as an efficient lock semaphore to facilitate
atomic operations. Rather than an actual pointer, this counting object's
address is often stored within a \texttt{uintptr\_t} variable; and then cast to
the correct pointer type (via \texttt{reinterpret\_cast}) when required. Only
the \texttt{reinterpret\_cast} presents a problem during constant evaluation;
and consequently our library-only libstdc++ implementation instead stores a
standard typed pointer to the counting object. At runtime, the casting then
occurs before the LSB toggle; while during constant evaluation there is no need
for locking, as this is a single threaded environment, and so no cast is
required.

\subsection{Two Memory Allocations}

Unlike \texttt{std::unique\_ptr}, a \texttt{std::shared\_ptr} must store not
only the managed object, but also the type-erased deleter and allocator, as
well as the number of \texttt{std::shared\_ptr}s and \texttt{std::weak\_ptr}s
which own or refer to the managed object. This information is managed as part
of a dynamically allocated object referred to as the \emph{control block}.

Existing runtime implementations of \texttt{std::make\_shared},
\texttt{std::allocate\_shared}, \linebreak
\texttt{std::make\_shared\_for\_overwrite}, and
\texttt{std::allocate\_shared\_for\_overwrite}, allocate memory for both the
control block, \emph{and} the managed object, from a single dynamic memory
allocation; via \texttt{reinterpret\_cast}.  This practise aligns with a remark
at \wgxxi{util.smartptr.shared.create}; quoted below:

\makebox[0pt][r]{(7.1)\quad} - Implementations should perform no more than one memory allocation.\newline
\makebox[0pt][r]{\quad} \phantom{-} [\emph{Note 1}: This provides efficiency equivalent to an intrusive smart pointer.  - \emph{end note}]

As \texttt{reinterpret\_cast} is not permitted within a constant expression, an
alternative approach is required for \texttt{std::make\_shared},
\texttt{std::allocate\_shared}, \texttt{std::make\_shared\_for\_overwrite},
and \linebreak \texttt{std::allocate\_shared\_for\_overwrite}.  A
straightforward solution is to create the object first, and pass its address to
the appropriate \texttt{std::shared\_ptr} constructor. Considering the control
block, this approach amounts to two dynamic memory allocations; albeit at
compile-time. Assuming that the runtime implementation need not change, the
remark quoted above can be left unchanged; as this is only a recommendation,
not a requirement.

\subsection{Relational Operators}

Comparing dynamically allocated pointers within a constant expression is legal,
provided the result of the comparison is not unspecified.  Such comparisons are
defined in terms of a partial order, applicable to pointers which either point
\emph{``to different elements of the same array, or to subobjects
thereof...''}; or to \emph{``different non-static data members of the same
object, or to subobjects of such members, recursively...''}; from paragraph 4
of \wgxxi{expr.rel}. A simple example program is shown below:

\begin{lstlisting}[language=cpp]
constexpr bool ptr_compare()
{
  int* p = new int[2]{};
  bool b = &p[0] < &p[1];
  delete [] p;
  return b;
}

static_assert(ptr_compare());
\end{lstlisting}

It is therefore unsurprising that we include the \texttt{std::shared\_ptr}
relational operators within the scope of our proposal to apply
\texttt{constexpr} to all functions within \wgxxi{smartptr}; the
\texttt{std::shared\_ptr} aliasing constructor makes this especially simple to
configure:

\begin{lstlisting}[language=cpp]
constexpr bool sptr_compare()
{
  double *arr = new double[2];
  std::shared_ptr p{&arr[0]}, q{p, p.get() + 1};
  return p < q;
}

static_assert(sptr_compare());
\end{lstlisting}

Furthermore, in the interests of \texttt{constexpr} consistency, we propose
that the relational operators of \texttt{std::unique\_ptr} \emph{also} now
include support for constant evaluation. As discussed above, the results of
such comparisons are very often well defined.

It may be argued that a \texttt{std::unique\_ptr} which is the sole owner of an
array, or an object with data members, presents less need for relational
operators. Yet we must consider that a custom deleter can easily change the
operational semantics; as demonstrated in the example below. A
\texttt{std::unique\_ptr} should also be legally comparable with itself.

\begin{lstlisting}[language=cpp]
constexpr bool uptr_compare()
{
  short* p = new short[2]{};
  auto del = [](short*){};
  std::unique_ptr<short[]>               a{p+0};
  std::unique_ptr<short[],decltype(del)> b{p+1, del};
  return a < b;
}

static_assert(uptr_compare());
\end{lstlisting}

\subsection{Smart Pointer Adaptors}
\label{sec:adapt}

C++23 introduced the smart pointer adaptors: \texttt{std::out\_ptr\_t} and
\texttt{std::inout\_ptr\_t}; as specified in \wgxxi{smartptr.adapt}. Objects of
these class templates adapt types such as smart pointers for functions that use
output pointer parameters. Smart pointer adaptors are intended for use only as
temporary objects, often created using the helper functions
\texttt{std::out\_ptr} or \texttt{std::inout\_ptr}, and typically created and
destroyed as arguments to a function call such as \texttt{f} below:

\begin{lstlisting}[language=cpp]
constexpr bool sptr_out_ptr()
{
  std::shared_ptr<int> sp;
  auto f = [&](void **pp) { *pp = new int{42}; };
  f(std::out_ptr(sp, std::default_delete<int>{}));
  return 42 == *sp;
}

static_assert(sptr_out_ptr());
\end{lstlisting}

The \texttt{f} function above expects a \texttt{void**} argument, and this is
obtained using the \texttt{operator void**} conversion operator member of the
\texttt{std::out\_ptr\_t} temporary; an \texttt{operator int**}
conversion operator is also available.

In runtime implementations, the pointer value returned by a conversion
operator, holds the address of a pointer reference member (e.g.
\texttt{\_M\_p}) of the smart pointer adaptor temporary. This member references
the smart pointer's stored pointer. A \texttt{reinterpret\_cast} may then be
used to obtain a \texttt{void**} from a \texttt{T**}; via an intermediate
\texttt{reinterpret\_cast} to a \texttt{void*}. The second cast, from
\texttt{void*} to \texttt{void**}, is not permitted during constant evaluation.

\begin{lstlisting}[language=cpp]
  _Tp*& _M_p;
#ifdef __cpp_lib_constexpr_shared_ptr
  void* _M_pv;
  void* _M_p_orig;
#endif
};
\end{lstlisting}

An implementation suitable for use during constant evaluation adds two
new \texttt{void*} data members; outlined above (\texttt{\_M\_pv} and
\texttt{\_M\_p\_orig}). Both pointers are initialised in the adaptor's
constructor to the smart pointer's stored pointer (\texttt{\_M\_p} in this
case). The address of the first (\texttt{\_M\_pv}) can be returned by the
\texttt{operator void**} conversion operator. The destructor can then update
\texttt{\_M\_p} if it is found that \texttt{\_M\_pv} differs from
\texttt{\_M\_p\_orig} (akin to a \texttt{bool} flag). The destructor for the
partial specialisation for raw (smart) pointers is shown below.

% Destructor for the partial specialization for raw pointers.
\begin{lstlisting}[language=cpp]
#ifdef __cpp_lib_constexpr_shared_ptr
constexpr
~_Impl()
{
  if (_M_pv != _M_p_orig)
    _M_p = static_cast<_Tp*>(_M_pv);
}
#endif
\end{lstlisting}

This approach aligns with the specified preconditions for the conversion
operators (\wgxxi{smartptr.adapt}), which states that only one of the two
conversion operators should be called on a single adaptor temporary.

A significant obstacle to this implementation strategy, is that it would cause
an ABI break for libstdc++; the addition of the two \texttt{void*} members
increasing the smart pointer adaptor object sizes, relative to those in the
current version. We do not request changes requiring such an implementation.
Consequently, we propose that of the two conversion operators within each of
the two pointer adaptors (\texttt{std::out\_ptr} and \texttt{std::inout\_ptr}),
only the typed (\texttt{Pointer*}) version should receive the
\texttt{constexpr} specifier; the \texttt{void**} version should remain as it
is.

\subsection{Maybe Not Now, But Soon}
\label{sec:mnnbs}

The functions from \wgxxi{smartptr} listed below cannot possibly be evaluated
within a constant expression. We \emph{do not} propose that their
specifications should change. While C++23's \cite{P2448R2} allows such
functions to be annotated as \texttt{constexpr}, we suggest that in this
instance the C++ community will be served better by a future update; when their
constant evaluation becomes possible.

\begin{itemize}
  \item \wgxxi{util.smartptr.hash}: The \texttt{operator()} member of the class template specialisations for \linebreak
\texttt{std::hash<std::unique\_ptr<T,D>}\texttt{>} and
\texttt{std::hash<std::shared\_ptr<T>}\texttt{>} cannot be defined according to the \emph{Cpp17Hash} requirements (\wgxxi{hash.requirements}). (A pointer cannot, during constant evaluation, be converted to an \texttt{std::size\_t} using \texttt{reinterpret\_cast}; or otherwise.)

  \item \wgxxi{util.smartptr.owner.hash}: The two \texttt{operator()} member functions of the recently adopted \texttt{owner\_hash} class, also cannot be defined according to the \emph{Cpp17Hash} requirements.

  \item \wgxxi{util.smartptr.shared.obs}: The recently adopted \texttt{owner\_hash()} member function of \texttt{std::shared\_ptr}, also cannot be defined according to the \emph{Cpp17Hash} requirements.

  \item \wgxxi{util.smartptr.weak.obs}: The recently adopted \texttt{owner\_hash()} member function of \texttt{std::weak\_ptr}, also cannot be defined according to the \emph{Cpp17Hash} requirements.

  \item \wgxxi{util.smartptr.shared.cast}: Neither of the two \texttt{reinterpret\_pointer\_cast} overloads can be included as their implementations will typically call \texttt{reinterpret\_cast}, which is prohibited here.

\end{itemize}

We also \emph{do not} propose any specification change for the overloads of
\texttt{operator<}\texttt{<} for \texttt{std::shared\_ptr} and
\texttt{std::unique\_ptr}, from \wgxxi{util.smartptr.shared.io} and
\wgxxi{unique.ptr.io}. Unlike the functions above, a \texttt{constexpr}
implementation for the overloads could today use a vendor-specific extension;
do nothing; or simply report an error. But such possibilities should be
discussed in a separate proposal focused on I/O.

\section{Impact on the Standard}

This proposal is a pure library extension, and does not require any new
language features.

\section{Implementation}

An implementation by the first author, based on libstdc++
can be found \href{https://github.com/SCT4SP/constexpr_shared_ptr}{here}. A
comprehensive test suite is included there within
\texttt{tests/shared\_ptr\_constexpr\_tests.cpp}; alongside a standalone bash
script to run it. All tests pass with recent GCC and Clang (i.e. versions
supporting \cite{P2738R1}; \texttt{\_\_cpp\_constexpr >= 202306L}).

A second implementation, by the second author, based on the ``libc++'' C++
Library is also available: on Github
\href{https://github.com/hanickadot/llvm-project/tree/P3309-constexpr-atomic-and-atomic-ref}{here}
(via commit 23217d0); and with a corresponding Compiler Explorer instance
\href{https://compiler-explorer.com/z/8cj7nc1no}{here}.

\section{Polls}

\subsection{LEWG, Hagenberg, 2025}

\textbf{Champion}: Hana Dusíková\newline
\textbf{Chair}: Andreas Weis / Inbal Levi\newline
\textbf{Minute Taker}: Guy Davidson\newline

\textbf{Summary}

\textbf{POLL}: Forward P3037R4 to LWG for C++26.

\begin{tabular}{|c|c|c|c|c|}
  \hline
  \textbf{SF} & \textbf{F} & \textbf{N} & \textbf{A} & \textbf{SA} \\
  \hline
  9 & 7 & 4 & 0 & 0  \\
  \hline
\end{tabular}

\textbf{Attendance}: 18 (IP) + 7 (R)\newline
\textbf{Author's Position}: SF (both)\newline
\textbf{Outcome}: Strong consensus in favour\newline

\section{Proposed Wording}
\label{sec:wording}

The following wording changes apply to the \emph{C++ Standard Draft} and can
also be viewed on Github via a fork of the \emph{C++ Standard Draft Sources}
repository
\href{https://github.com/cplusplus/draft/compare/\upstreamcommit...pkeir:draft:\forkcommit}{here}.

Pages from the \emph{C++ Standard Draft Sources} which include wording changes
proposed here are also provided on the following pages; with changes
highlighted.

% As includepdf loses the highlighting/strikeout annotations
% pdftk A=P3037R5.pdf B=std-diff.pdf cat A1-6 B A33-end output output.pdf
% https://askubuntu.com/a/18691
% Yes, this is using std-diff.pdf *twice*; the first time below ensures
% contents page numbers are correct.
\includepdf[pages=-]{std-diff.pdf}

\section{Acknowledgements}

Thanks to all of the following:

  - (In alphabetical order by last name.) Peter Dimov, Thiago Macieira, Arthur O'Dwyer, Jonathan Wakely and everyone else who contributed to the BSI Panel, SG7, LEWG, LWG, and online forum discussions.

\section{References}
\renewcommand{\section}[2]{}%
\begin{thebibliography}{9}

  \bibitem[P0784R7]{P0784R7}
    Peter Dimov, Louis Dionne, Nina Ranns, Richard Smith, Daveed Vandevoorde,
    \emph{More constexpr containers} (2019)\newline
    \url{https://wg21.link/P0784R7}

  \bibitem[P2738R1]{P2738R1}
    Corentin Jabot, David Ledger,
    \emph{\texttt{constexpr} cast from \texttt{void*}: towards \texttt{constexpr} type-erasure} (2023)\newline
    \url{https://wg21.link/P2738R1}

  \bibitem[P2448R2]{P2448R2}
    Barry Revzin,
    \emph{Relaxing some \texttt{constexpr} restrictions} (2022)\newline
    \url{https://wg21.link/P2448R2}

  \bibitem[P2273R3]{P2273R3}
    Andreas Fertig,
    \emph{Making \texttt{std::unique\_ptr} constexpr} (2021)\newline
    \url{https://wg21.link/P2273R3}

  \bibitem[P3309R3]{P3309R3}
    Hana Dusíková,
    \emph{constexpr atomic<T> and atomic\_ref<T>} (2024)\newline
    \url{https://wg21.link/P3309R3}

  \bibitem[P3068R6]{P3068R6}
    Hana Dusíková,
    \emph{Allowing exception throwing in constant-evaluation} (2024)\newline
    \url{https://wg21.link/P3068R6}

  \bibitem[ClangOz]{ClangOz}
    Paul Keir, Andrew Gozillon,
    \emph{ClangOz: Parallel constant evaluation of C++ map and reduce operations} (2024)\newline
    \url{https://doi.org/10.1016/j.cola.2024.101298}

\end{thebibliography}

\end{document}
